%%%%%%%%%%%%%%%%%%%%%%%%%%%%%%%%%%%%%%%%%
% Medium Length Professional CV
% LaTeX Template
% Version 2.0 (8/5/13)
%
% This template has been downloaded from:
% http://www.LaTeXTemplates.com
%
% Original author:
% Trey Hunner (http://www.treyhunner.com/)
%
% Important note:
% This template requires the resume.cls file to be in the same directory as the
% .tex file. The resume.cls file provides the resume style used for structuring the
% document.
%
%%%%%%%%%%%%%%%%%%%%%%%%%%%%%%%%%%%%%%%%%

%----------------------------------------------------------------------------------------
%	PACKAGES AND OTHER DOCUMENT CONFIGURATIONS
%----------------------------------------------------------------------------------------

\documentclass{resume} % Use the custom resume.cls style

\usepackage[left=0.75in,top=0.6in,right=0.75in,bottom=0.6in]{geometry} % Document margins

\name{Zeeshan Rafique} % Your name
\address{Karachi, Pakistan \\ zeeshanrafique.me \\ (+92)~$\cdot$~344 2350520 \\ zeeshanrafique23@gmail.com} % Your phone number and email

\begin{document}

%----------------------------------------------------------------------------------------
%	EDUCATION SECTION
%----------------------------------------------------------------------------------------
\begin{rSection}{Education}

{\bf Usman Institute of Technology (NED), Karachi} \hfill {\em July 2021} \\ 
B.E. in Computer Engineering {\em CGPA: 3.4}

\end{rSection}

%----------------------------------------------------------------------------------------
%	WORK EXPERIENCE SECTION
%----------------------------------------------------------------------------------------
\begin{rSection}{Experience}

%----------------------Xcelerium-------------------------
\begin{rSubsection}{Xcelerium}{Nov 2021 - Present (Part-time)}
{Hardware Engineer}{Irvine, CA (Remote)}

\item Responsible of doing co-simulations \& AFI generation to emulate RISC-V System-on-Chip (SoC) on AWS-FPGA.
\item Vectorized Floating Point Unit (FPU) for high bandwidth data for RISC-V vector SoC.
\item Redesigned existing non-multi-precision FPU to support multi-precision operations to increase resource sharing.
\end{rSubsection}

%-----------------------MERL-------------------------
\begin{rSubsection}{Micro Electronics Research Lab (MERL)}{Oct 2019 - Present}
{Hardware Researcher}{Karachi, PK}
\item Worked on designing and verification of embedded RISC-V processors.
\item Involved in the complete development cycle of SoC from designing to chip bring-up.
\item Part of a team who has collectively done 6 tape-outs on SKY130nm open-source PDK with the addition of new features in each revision. Also, done one commercial SoC tape-out on TSMC65nm PDK.
\item Managed Git organization and repositories. Also introduced automation in workflows.
\end{rSubsection}

%------------------Google Summer of Code------------------
\begin{rSubsection}{Google Summer of Code}{Jun 2021 - Aug 2021}
{Contributor}{Remote}
\item Worked on a RISC-V CPU called SERV to add support for Multiplication and Division operations to optimize the execution of the programs having Mul/Div instructions. 
\item The project was completed in 3 months along with documentation.
    
\end{rSubsection}

\end{rSection}

%----------------------------------------------------------------------------------------
%	PROJECTS SECTION
%----------------------------------------------------------------------------------------

\begin{rSection}{Projects}

%------------------------------------------------
\begin{rSubsection}{Azadi-SoC}{https://github.com/merledu/azadi-soc}
{SystemVerilog, Tcl, Python, Makefile, RV GCC}{MERL}
Azadi-SoC is an embedded low-power 32-bit SoC that integrates a RISC-V processor with peripherals using Tilelink(UL) bus. I was involved in the RTL front-end of this SoC, system-level verification, emulation on FPGA, and in Synthesis flow to hand over a clean netlist to the back-end team. We taped out this SoC on TSMC-65nm.

\end{rSubsection}

%------------------------------------------------
\begin{rSubsection}{Multi-precision FPU}{}
{SystemVerilog, RISC-V, Cocotb}{Xcelerium}
Rewrote the existing FPU modules to support multi-precision and to increase resource sharing. Also, designed a wrapper for vectorized floating point unit to handle a high bandwidth input data stream. The unit testing was performed using Cocotb (Python lib.) based test benches.

\end{rSubsection}

%------------------------------------------------
\begin{rSubsection}{Buraq-mini-RV}{https://github.com/merledu/Buraq-mini}
{SystemVerilog, RISC-V, C++}{MERL}
A 32-bit 5-stage pipelined core designed in SystemVerilog. It implements the RISC-V base ISA and multiply/divide instructions. Basic assembly tests were executed on the core for verification. This project was intended for learning the execution of instructions in a pipelined environment. 

\end{rSubsection}
    
\end{rSection}

%----------------------------------------------------------------------------------------
%	TECHNICAL STRENGTHS SECTION
%----------------------------------------------------------------------------------------

\begin{rSection}{Technical Strengths}

\begin{tabular}{ @{} >{\bfseries}l @{\hspace{5ex}} l }
Hardware Descriptive Languages & SystemVerilog, Verilog, CHISEL\\
Interconnect \& Peripheral Protocols & AXI4, APB, TileLink, SPI, UART, PWM \\
Computer Languages & Python, C/C++, Tcl, Shell scripting,  RISC-V Assembly \\
Concepts & RISC-V ISA, MMIOs, STA, Synthesis \\
Tools & Cadence(Xcelium, Genus), Verilator, Xilinx- \\ & Vivado, Git, Linux, Vim, RV GCC Compiler
\end{tabular}

\end{rSection}

\end{document}
